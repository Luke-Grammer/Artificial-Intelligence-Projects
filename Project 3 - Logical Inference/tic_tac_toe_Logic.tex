\documentclass[12pt]{article}
    \usepackage{amsmath,amssymb,amsthm,latexsym,geometry}
    \geometry{ margin=0.7in}
    \newcommand\tab[1][3cm]{\hspace*{#1}}
    
\begin{document}
\vspace*{-10mm}

{\large \noindent
\underline{Tic-Tac-Toe Strategy:}} \\
\\
The strategy used for tic-tac-toe follows eight steps in decreasing precedence. \\
\\
\textbf{1)} If a winning move can be made, make it. \\
\tab\textbf{I.} Check rows \\
\begin{center}
\begin{tabular}{c|c|c}
	  &   &   \\      \hline
	O & O &(O)\\      \hline
	  &   &  
  \end{tabular} \\
\end{center}
\tab\textbf{II.} Check columns \\
\begin{center}
	\begin{tabular}{c|c|c}
		  & O &   \\      \hline
		  &(O)&   \\      \hline
		  & O &  
	  \end{tabular} \\
	\end{center}
\tab\textbf{III.} Check Diagonals \\
\begin{center}
	\begin{tabular}{c|c|c}
		O &   &  \\      \hline
		  & O &  \\      \hline
		  &   &(O)
	  \end{tabular} \\
	\end{center}
~\\
\textbf{2)} If an opponent can make a winning move, block it. \\
\tab\textbf{I.} Block rows \\
\begin{center}
\begin{tabular}{c|c|c}
	  &   &   \\      \hline
	O & O & (X)  \\      \hline
	  &   &  
  \end{tabular} \\
\end{center}
\tab\textbf{II.} Block columns \\
\begin{center}
	\begin{tabular}{c|c|c}
		  & O &   \\      \hline
		  &(X)&   \\      \hline
		  & O &  
	  \end{tabular} \\
	\end{center}
\tab\textbf{III.} Block Diagonals \\
\begin{center}
	\begin{tabular}{c|c|c}
		O &   &  \\      \hline
		  & O &  \\      \hline
		  &   &(X)
	  \end{tabular} \\
	\end{center}
\pagebreak 
\textbf{3)} If a move can set up a guaranteed win next turn, make that move. \\
\tab\textbf{I.} If one piece is in an otherwise empty row, and another piece is in an otherwise 
	\tab\;\;\;\;empty column, place a piece in the intersection if it's blank \\
\begin{center}
\begin{tabular}{c|c|c}
	X & O &   \\      \hline
	O &(O)&   \\      \hline
	  &   &  
  \end{tabular} \\
\end{center}
\tab\textbf{II.} If two pieces are on the ends of a row or column, and the center and adjacent 
	\tab\;\;\;\;\;\;side is free, place a piece on the other end of either diagonal \\
\begin{center}
	\begin{tabular}{c|c|c}
		O & X & O \\      \hline
		  &   &   \\      \hline
		  &   &(O) 
	  \end{tabular} \\
	\end{center}
~\\
\textbf{4)} If an opponent can set up a guaranteed win next turn, block that move. \\
\tab\textbf{I.} If one opponent piece is in an otherwise empty row, and another opponent piece 
	\tab\;\;\;\;is in an otherwise empty column, place a piece in the intersection if it's blank \\
\begin{center}
\begin{tabular}{c|c|c}
	X & O &   \\      \hline
	O &(X)&   \\      \hline
	  &   &  
  \end{tabular} \\
\end{center}
\tab\textbf{II.} If two opponent pieces are on the ends of a row or column, and the center and 
\tab\;\;\;\;\;\;adjacent side is free, place a piece on the other end of either diagonal \\
\begin{center}
	\begin{tabular}{c|c|c}
		O & X & O \\      \hline
		  &   &   \\      \hline
		  &   &(X) 
	  \end{tabular} \\
	\end{center}
~\\
\textbf{5)} If the center is available, move into the center. \\
\\
\textbf{6)} If an opponent is in the corner, move into an adjacent corner if possible. \\
\\
\textbf{7)} If any corners are available, move into a corner. \\
\\
\textbf{8)} If any sides are available, move into a side. \\
 \end{document}